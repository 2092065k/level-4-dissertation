\documentclass{l4proj}

\usepackage{url}
\usepackage{fancyvrb}
\usepackage[final]{pdfpages}


\begin{document}
\title{Implementations of Advanced K-Means Clustering Algorithms \\ for Spark/MLlib}
\author{Ivan Kyosev}
\date{March 21, 2017}
\maketitle

\begin{abstract}
Apache Spark is a fast and general engine for large-scale data processing. It emphasizes speed, ease of use and generality in regards to working with Big Data. Its feature-set provides an alternative to the popular Hadoop MapReduce software framework, displaying a noticeable performance improvement. One of Spark's many components is MLlib. It is a machine learning library built on top of the core features of Spark and provides implementations for a variety of popular machine learning algorithms, written in a way meant to support scalability when operating on big data sets.

This paper presents a proposed extension to the MLlib component of Apache Spark in the form of additional K-Means Clustering algorithms that the library does not currently support. K-Means clustering is a method of vector quantization, originally from signal processing, that is popular for cluster analysis in data mining. A thorough analysis of the performance of the newly created algorithms and the relative quality of the produced data models is also conducted.
\end{abstract}

\educationalconsent

\tableofcontents

\chapter{Introduction}
\label{intro}
\pagenumbering{arabic}

intro text

\section{Aims}

aims text

\section{Report outline}

The remainder of this report will focus on the mathematics underlying each of the considered K-Means clustering algorithms as well as the challenges of implementing them in a scalable fashion under Spark:
\begin{itemize}
\item Chapter~\ref{spark} covers the data processing model provided by Apache Spark and its machine learning library MLlib
\item Chapter~\ref{kmeans} explains the most common used versions of K-Means Clustering algorithms
\item Chapter~\ref{previous} discusses the approach taken by Spark's developers to create the current clustering API
\item Chapter~\ref{propose} outlines the proposed algorithms to be implemented as part of Apache Sparks, along with their benefits and challenges
\item Chapter~\ref{online} details two different approaches to the implementation of Lloyd's Online K-Means Clustering algorithm
\item Chapter~\ref{art} presents a version of K-Means using Adaptive Resonance Theory and its implementation
\item Chapter~\ref{som} presents the implementation of a  a K-Means Clustering algorithm using Self-Organizing Maps
\item Chapter~\ref{eval} covers a variety of ways to evaluate both the performance of the presented algorithms and the quality of the data models they produce
\item Chapter~\ref{conclusion} Concludes this report with a reflection on the whole project
\end{itemize}

%==============================================================================

\chapter{Introduction to Apache Spark and MLlib}
\label{spark}
\section{Apache Spark}

spark text

\section{Resilient Distributed Datasets}

RDD text

\section{MLlib}

MLlib text

%==============================================================================

\chapter{K-Means Clustering}
\label{kmeans}

top level k-means text?

\section{Iterative K-Means Clustering}

iterative k-means text

\section{Mini-batch K-Means Clustering}

mini-batch text

%==============================================================================

\chapter{Previous Work in MLlib}
\label{previous}

\section{Implementation of Iterative K-Means Clustering}

\section{Implementation of Mini-batch K-Means Clustering}

%==============================================================================

\chapter{Proposed Extension to Spark/MLlib}
\label{propose}

\section{Online K-Means}

\section{Adaptive Resonance Theory K-Means}

\section{Self-Organizing Maps K-Means}

\section{Benefits of Online Algorithms}

\section{Implementation Challenge in Parallelizing Sequential Algorithms}

%==============================================================================

\chapter{Online K-Means}
\label{online}

\section{Implementation Using Grand Mean}

\section{Implementation using Spark Streaming}

%==============================================================================

\chapter{Adaptive Resonance Theory K-Means}
\label{art}

\section{Implementation Using Spark Streaming}

%==============================================================================

\chapter{Self-Organizing Maps K-Means}
\label{som}

\section{Implementation Using Spark Streaming}

%==============================================================================

\chapter{Performance Evaluation and Data Model Analysis}
\label{eval}

maybe add an evaluation section earlier to each part\\
add chapter comparing algorithms at the end\\
both would need to explain evaluation method earlier\\
make a reference to the used training data

\section{Sum of Squared Distances}

\section{Silhouette}

%==============================================================================

\chapter{Conclusion}
\label{conclusion}

conclusion text - multiple sections?

%==============================================================================

\bibliographystyle{plain}
\bibliography{l4proj}
\end{document}
