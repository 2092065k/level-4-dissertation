\documentclass{l4proj}

\usepackage{url}
\usepackage{fancyvrb}
\usepackage[final]{pdfpages}
\usepackage{hyperref}
\usepackage{fancyvrb}
\usepackage{float}
\usepackage{multirow}
\usepackage{graphicx}
\usepackage{amsmath}
\usepackage{MnSymbol}
\usepackage{wasysym}
\usepackage{mathtools}
\usepackage{subcaption}
\usepackage{relsize}
\usepackage{algorithm}
\usepackage[noend]{algpseudocode}

\hypersetup{
    colorlinks,
    citecolor=black,
    filecolor=black,
    linkcolor=black,
    urlcolor=black
}

\makeatletter
\def\BState{\State\hskip-\ALG@thistlm}
\makeatother

\begin{document}
\title{Implementations of Advanced K-Means Clustering Algorithms \\ for Spark/MLlib}
\author{Ivan Kyosev}
\date{March 21, 2017}
\maketitle

\begin{abstract}
Apache Spark is a fast and general engine for large-scale data processing. It emphasizes speed, ease of use and generality in regards to working with Big Data. Its feature-set provides an alternative to the popular Hadoop MapReduce software framework, displaying a noticeable performance improvement. One of Spark's many components is MLlib. It is a machine learning library built on top of the core features of Spark and provides implementations for a variety of popular machine learning algorithms, written in a way meant to support scalability when operating on big data sets.

This paper presents a proposed extension to the MLlib component of Apache Spark in the form of additional K-Means Clustering algorithms that the library does not currently support. K-Means clustering is a method of vector quantization, originally from signal processing, that is popular for cluster analysis in data mining. A thorough analysis of the performance of the newly created algorithms and the relative quality of the produced data models is also conducted.
\end{abstract}

\educationalconsent

\tableofcontents

\chapter{Introduction}
\label{intro}
\pagenumbering{arabic}

Machine learning is an increasingly popular and relevant part of modern day systems looking to extract knowledge and value from the world around them. It refers to an approach to computation, where applications can learn without being explicitly programmed. These kinds of apps can analyze a provided sample of data and make predictions or decisions, based on some form of discovered pattern or correlation between the individual elements of the input set.

By being able to take rational choices, through examining the surrounding environment, systems that employ machine learning have become a central component of artificial intelligence. Because these system adapt their behavior based on new ``experiences'' they are utilized in a wide range of computing tasks, where developing explicit algorithms is not feasible. Such examples include solving problems in speech recognition, vision and robotics.

These machine learning algorithms are typically classified into three different categories based on the nature of the data they are learning from and the feedback of information for each input item:

\begin{itemize}
\item Supervised learning - in this case a program presented with a particular input set of data is also provided with the desired output, with the end goal of having the machine learn a mapping function from inputs to outputs
\item Unsupervised learning - here the elements of the input data set are not provided with a desired output value in pairs, so the system is left on its own to find structure in its input
\item Reinforcement learning - this refers to the idea of having an program interact with a dynamic environment, where it is attempting to achieve a certain goal. Along with every action, the program is also provided with positive or negative feedback, depending on whether the action aided in achieving the goal or not. 
\end{itemize}

Many of these machine learning algorithms have a similar underlying approach to their implementation. Input training data with some unknown probability distribution is presented and the program has to construct a model that allows it to extract meaning from the data and make sufficiently accurate predictions for instances that have not been encountered yet. This is generally achieved by defining a loss function and adjusting the model in attempts to minimize this loss with respect to the inputs -- effectively an optimization problem.

It is clear that machine learning algorithms heavily depend on the availability of input data in order to optimize their models to more accurately reflect the state of the real world. Therefore the quantity of supplied input data begins to have a significant impact on the quality of the output produced by application relying on machine learning. With the advent of big data and technologies meant to enable fast and reliable processing of big data, these algorithms can now make use of vast amounts of information, enabling them to create more accurate and valuable results. Examples of big data processing frameworks include Hadoop and Apache Spark.

The Apache Spark project in particular includes a component called MLlib. MLlib is a machine learning library built on top of the core features of Spark. It provides implementations for a variety of popular machine learning algorithms, all of which are written in a way meant to support horizontal scalability\footnote{Horizontal scaling refers to the scaling of a system by adding more parallel, similar in performance nodes. It is directly contrasted by vertical scaling where additional resources are added to the one node of a system.} when operating on big data sets.

\section{Aims}

The aim of this project is to create an extension to the MLlib component of Apache Spark, by developing implementations for a variety of K-Means clustering algorithms that the framework does not currently support. The newly introduced algorithms also need to be evaluated and contrasted with the ones already provided by Spark.

K-Means clustering is an example of an unsupervised machine learning algorithm. It is a method of vector quantization\footnote{Quantization is the process of mapping a large set of input values to a smaller set.}, that originated from signal processing. Its primary purpose is to group a collection of \texttt{N} data points into \texttt{K} distinct clusters. There is a variety of algorithms that produce the end result of grouping data points into \texttt{K} clusters, with varying levels of speed and accuracy. The ones that this project aims to implement as part of Spark could deliver a desired performance improvement with minimal loss in quality.

\section{Report outline}

The remainder of this report will focus on the mathematics underlying each of the considered K-Means clustering algorithms as well as the challenges of implementing them in a scalable fashion under Spark:
\begin{itemize}
\item \textbf{Chapter~\ref{spark}} covers the data processing model provided by Apache Spark and its machine learning library MLlib
\item \textbf{Chapter~\ref{kmeans}} explains the most common used versions of K-Means Clustering algorithms
\item \textbf{Chapter~\ref{previous}} discusses the approach taken by Spark's developers to create the current clustering API
\item \textbf{Chapter~\ref{propose}} outlines the proposed algorithms to be implemented as part of Apache Sparks, along with their benefits and challenges
\item \textbf{Chapter~\ref{online}} details two different approaches to the implementation of Lloyd's Online K-Means Clustering algorithm
\item \textbf{Chapter~\ref{art}} presents a version of K-Means using Adaptive Resonance Theory and its implementation
\item \textbf{Chapter~\ref{som}} presents the implementation of a  a K-Means Clustering algorithm using Self-Organizing Maps
\item \textbf{Chapter~\ref{eval}} covers a variety of ways to evaluate both the performance of the presented algorithms and the quality of the data models they produce
\item \textbf{Chapter~\ref{conclusion}} Concludes this report with a reflection on the whole project
\end{itemize}

%==============================================================================

\chapter{Introduction to Apache Spark and MLlib}
\label{spark}
\section{Apache Spark}

A new big data processing paradigm in the form of cluster computing has become widely popular in recent years. It is a model where data-parallel computations are executed on clusters of unreliable machines by systems that provide scheduling, fault tolerance and load balancing\cite{Spark}. Hadoop MapReduce was the first software framework to implement these principles\cite{MapReduce}, with systems like Dryad and Map-Reduce-Merge generalizing the supported types of data flows. These system achieve fault tolerance and scalability by supplying the user with a means of expressing computation as an acyclic data flow graph which passes the input data through a set of operations. This enables the underlying system to handle any occurring faults as well as scheduling without any user interaction.

Apache Spark is one of the latest big data processing frameworks which implements the cluster computing model. One of its primary advantages over Hadoop is that it can perform in-memory data processing, while Map-Reduce persists all outputs back to disk after a map or reduce action. This extends the scope of problems which can be solved efficiently by Spark to also include applications that reuse a working set of data across multiple parallel operations. It is impractical to implement these kinds of apps under Hadoop as between each iteration of an algorithm over the working set, the job must reload the data from disk, causing unnecessary overhead, resulting in degraded performance. Examples include iterative jobs and interactive analysis\cite{Spark}.

Due to its differences in design philosophy, users of the Spark framework have reported a 100x speed increase in certain work loads, versus a similar MapReduce solution, as well as a 10x faster execution on disk\cite{webSpark}. This performance improvement, however, results in the system requiring more RAM to carry out its operations.

The core components of Spark are implemented in Scala - a statically typed functional programming language that runs on top of the Java Virtual Machine. It also includes object-oriented features and supports Java interoperability thanks to its JVM roots -- this allows Spark to easily interact with Java based systems such as HDFS. In addition to Scala the framework also export an API available in Java, Python and R.

Spark can also be used interactively from a modified version of the Scala interpreter. This method provides full access to the API, allowing a user to define variables, functions, classes and run parallel jobs on a computer cluster. Spark is possibly the first system that allows a general-purpose programming language to be used interactively to process large data sets on a cluster\cite{Spark}.

\section{Resilient Distributed Datasets}

To achieve its goals, Spark introduces an abstraction called resilient distributed datasets -- RDDs. They are a parallel, fault-tolerant data structure that lets users explicitly persist intermediate results in memory, control their partitioning to optimize data placement, and manipulate them using a rich set of operations\cite{RDD}. The RDD API is the primary way of expressing computation on big data sets in Apache Spark.

\subsection{Basics of the RDD API}

An RDD is a read-only partitioned collection of records. It supports a series of methods that resemble operations, available to most collections in a functional programming language\footnote{Despite the nature of the RDD based API, code written in Spark does not need to adhere to the principles of pure functional programing.}. RDDs can only be created through deterministic operations on either data in stable storage or other RDDS. These kind of functions are known as transformations -- examples include \textit{map}, \textit{filter}, \textit{flatMap} and others. Transformations are lazy\footnote{A lazy function is not evaluated when it is called, but rather when something else tries to access its result.} operations that define a new RDD, which is typically the result of applying the same function to every data item in the source RDD -- making Spark ideal for batch processing. By chaining together several of these transformations a user can define a directed acyclic graph (DAC) of operations the data needs to pass through (achieved in a parallel fashion).

Along with transformations, the RDD API also supports actions. These are operations that return a value to the application or export data to a storage system. Examples of actions include - \textit{count}, \textit{collect}, \textit{reduce} and others. Typically a user would define a DAC of transformations to be performed on some data source, these operations would then not be executed until an action is invoked, at which point the result of all transformations in the DAC is evaluated along with the specified action at the end. If there is a large sequence of transformations within the DAC of an RDD, that will be operated on in the end by multiple different actions (branching towards a later point in the DAC), a developer might choose to checkpoint an intermediate stage of the RDD, so that Spark does not need to recompute every operation applied to that RDD.

\begin{table}[H]
\label{Actions and Transformations}
\caption{Transformations and actions available on RDDs in Spark.}
\includegraphics[width=1.0\textwidth]{images/spark-trans-action-edited}
\end{table}

\subsection{RDD fault-tolerance}

To begin the processing of an RDD, the individual data partitions (making up the source used to create the RDD -- typically a file in HDFS), are sent to different nodes in the computing cluster where the user defined DAC of operations is executed simultaneously for each partition (if the system could afford to allocate enough resources for a one-to-one mapping of partitions to cluster nodes). In a large scale distributed system, however, failure of individual hardware components is to be expected. When one of the clusters fails, while processing an RDD partition, Spark is able to reconstruct the lost data without notifying the user.

Because RDDs provide an interface based on transformations (e.g. map, flatMap) that apply the same operations to many data items, they can efficiently provide fault-tolerance by logging the transformations used to build a dataset rather than the actual data (although checkpointing RDDs can have its advantages as previously mentioned). This log of operations is also known as the lineage of the RDD. Thus if a partition of an RDD is lost, the RDD has enough information about how it was derived from other RDDs to recompute just the missing partition\cite{RDD}. Therefore lost data can be quickly recovered without the need for costly replication.

The lineage of an RDD is a powerful property, since it ensures that a program cannot reference an RDD that it cannot reconstruct after failure -- it can always recompute all partitions from data in stable storage.

\subsection{RDD representation}

The Apache Spark framework provides a graph-based representation of RDDs where each dataset has a common interface, exposing five pieces of information:

\begin{itemize}
\renewcommand{\labelitemi}{\scriptsize$\blacksquare$}
\item a set of partitions, which are atomic pieces of the dataset
\item a set of dependencies on parent RDDs
\item a function for computing the dataset based on its parents
\item metadata about its partition scheme
\item metadata about its data placement
\end{itemize}

An RDD representing an HDFS file, for example, has as many partitions as the file has blocks and knows on which machine each block is on. If a map transformation were to be applied on this RDD, the resulting dataset would have the same partitions, but the map function would be applied on the parent's data when computing its elements.

An important aspect of this common interface is the representation of dependencies between RDDs. They are classified into two types: \textit{narrow} dependencies, where each partition of the parent RDD is used by at most one partition of the child RDD, and \textit{wide} dependencies where multiple child partitions may depend on a single parent partition\cite{RDD}. The \textit{map} operation would be an example of a narrow dependency, where as \textit{groupByKey} would result in a wide dependency.

The distinction between these two forms of dependencies has two primary implications. First, narrow dependencies allow for pipelined execution on one cluster node. This means that multiple successive operations resulting in narrow dependencies can be carried out on an RDD partition in the same computing cluster, without needing to move data between cluster nodes. Wide dependencies, however, require that data from all parent partitions to be available and to be shuffled across the nodes of the cluster. Second, narrow dependencies allow for a more efficient recovery from failure, as only the lost partitions need to be recomputed -- this can also be done in parallel on different processing nodes. When there are wide dependencies in the lineage graph of an RDD, however, a single failed node could loose information derived from all parent partitions, requiring a full re-execution of all transformations in the DAC, operating on the RDD.

\begin{figure}[H]
\label{RDD dependencies}
\caption{Examples  of  narrow  and  wide  dependencies.  Each box is an RDD, with partitions shown as shaded rectangles.}
\includegraphics[width=1.0\textwidth]{images/rdd_dependency}
\end{figure}

\section{Shared Variables}

When invoking transformations on an RDD -- such as \textit{map} and \textit{filter}, developers pass closures (functions) to Spark. Similar to functional programming these closures can refer to any variable in the scope where they are created. Once Spark begins the execution of those transformations on a worker node, the referenced local variables are copied to the worker. However, Spark also allows programmers to create two restricted types of shared variables to support common usage patterns\cite{Spark}.

If a Spark program contains a large read-only piece of data that is used in multiple parallel operations, it is preferable to distribute it to the workers only once, as opposed to including it with each closure. This is accomplished by creating a ``broadcast variable'' object which acts as a wrapper for the desired value. It ensures that it is only copied once to each worker, thus reducing network traffic and improving performance.

The second type of shared variables provided by Spark are known as ``accumulators''. Workers can only ``add'' to these variables by using an associative operation, and their value can only be read by the driver program. They enable easy implementations of counters (similar to MapReduce) and provide a more imperative syntax for parallel sums. Accumulators can be defined for any data type that has a ``zero value'' and an ``add'' operation.

\section{Spark Streaming}

In addition to its base API, Spark also provides a solution for developing applications based on sacalble, high-throughput, fault-tolerant stream processing of live data. This is achieved thanks to Spark Streaming - a system that allows data to be ingested from many sources like Kafka, Flume, Kinesis, or TCP sockets, and later processed using complex algorithms expressed with high-level functions.

Once a Spark program defines a input data source, Spark Streaming can receive a live data stream from the specified source, which is then divided into batches that are later processed by the Spark engine. The primary abstraction that implements this operation is the \textit{discretized stream} or \textit{DStream}\cite{DStream}. It is represented internally as a sequence of RDDs.

Once a DStream has been created from a specified live data source (e.g. Kafka, an HDFS directory, whose content is being simultaneously altered by a different application, etc.) or alternatively from a different DStream, it begins to store or log any new information it has received. A program utilizing Spark Streaming needs to also specify a batch interval to the streaming context -- used to configure all DStreams for that application. At the end of every interval the DStream produces an RDD, containing all the data output from the specified source, since the start of the interval. The resulting RDDs are then processed by the remaining application logic -- various transformations and actions. Finally, the batch interval must be set based on the latency requirements of  the application and the available cluster resources.

\section{MLlib}

The Apache Spark project also includes a distributed machine learning library, built on top of Spark's core features -- MLlib. MLlib provides efficient functionality for a wide range of learning settings and includes several underlying statistical, optimization,
and linear algebra primitives. It exploits the benefits of Spark's fault-tolerant, parallel big data processing paradigm to deliver fast and scalable implementations of standard learning algorithms for common learning settings including:  classification, regression, collaborative filtering, clustering, and dimensionality reduction\cite{MLlib}.

MLlib also inter-operates well with Spark's other major components. Thanks to Spark SQL\cite{SQL} developers can more easily designate their prior storage solutions as the data source of the desired machine learning algorithms. Spark Streaming is also used by MLlib to enable applications to learn from real-time online streamed data, however, this is only supported by certain algorithms.

%==============================================================================

\chapter{K-Means Clustering}
\label{kmeans}

A long time pursuit in the field of artificial intelligence and machine learning has been to grant machines the ability to differentiate between discrete objects with similar or vastly different properties and to be able to identify and classify them. At a high level, an AI would need to be able to examine the features of an object, which has not been previously encountered and create an internal, digital representation of it. Then the machine would need to perform a computation on that model to extract meaning from it and predict the nature of the object, without human input - i.e. explicitly providing a label.

Recent developments in image processing and robotics have had a great impact on the field of machine learning, in regards to allowing systems that employ these practices to interact more directly in the real world. The aim of this paper, however, is not to cover techniques for creating a mapping from real world objects to internal, machine friendly representations of their properties, but rather to outline the implementation of algorithms for operating on said representations -- focusing on cluster analysis.

Cluster analysis is concerned with the task of grouping a set of objects in a way, such that objects in the same group (known as a cluster) are more similar to each other than those in other groups. It is commonly used in statistical data analysis as well as data mining. Along with machine learning, cluster analysis has applications in other fields such as pattern recognition, information retrieval, biometrics and others.

In cluster analysis, different approaches to the problem of groping data together can vary greatly -- algorithms can differ in their notation regarding what constitutes a cluster, as well as how to find them. Popular definitions of clusters include: groups with small distances between the individual cluster members, dense areas of the data space or specific statistical distributions. Given a way of evaluating the quality of a particular assignment of individuals to groups, clustering can also be described as an optimization problem, were we are trying to maximize that value.

K-Means clustering is an examples of cluster analysis with a centroid-based model. It was originally used in signal processing as a method for vector quantization. K-Means functions by partitioning \texttt{N} observable objects into \texttt{K} distinct clusters, where every object belongs to the cluster with the nearest mean  -- here these observations will only take the form of N-dimensional data points. The actual clusters are represented as a central vector (N-dimensional point), which does not necessarily belong to the input set.

\begin{figure}[H]
\label{basic clusters}
\centering
\begin{subfigure}{.5\textwidth}
  \centering
  \includegraphics[width=1.2\linewidth]{images/figure_1}
  \label{fig:sub1}
\end{subfigure}%
\begin{subfigure}{.5\textwidth}
  \centering
  \includegraphics[width=1.2\linewidth]{images/figure_2}
  \label{fig:sub2}
\end{subfigure}
\caption{Examples of three distinct clusters in 3 dimensional space}
\label{fig:test}
\end{figure}

When the number of centroids \texttt{K} is fixed, K-Means clustering provides a formal definition of an optimization problem: find the \texttt{K} cluster centers and assign the input data points to the nearest center, such that the sum of squared distances from the observed points to their respective centers is minimized. This is expressed by the following formula:

\begin{figure}[H]
\label{wcss}
\[ \scalebox{2}{$ arg_{s} min \sum_{i=1}^{k}\sum_{x \in S} \|x - \mu_{i} \|^{2} $} \]
\caption{Within-Cluster Sum of Squares (WCSS)}
\end{figure}

Here $x_{1}$, $x_{2}$ ... $x_{n}$ is the set of observations in the form of multi-dimensional data points, $S_{1}$, $S_{2}$ .. $S_{k}$ are the \texttt{K} partitions of data points that are assigned to the \texttt{K} centers, and $\mu_{i}$ is the value of the centroid vector for partition $S_{i}$. This value is also referred to as the within-cluster sum of squares (WCSS) or the loss for the particular clustering.

The K-Means clustering optimization problem is also known to be NP-hard, therefore  a common approach is to simply search for approximate solutions. There is a number of efficient heuristic algorithms that quickly converge to a local optimum. Two of the most frequently used ones are Lloyd's algorithm (called Iterative K-Means here due to its reliance on iterative processing of the input data) and Mini-batch K-Means. These two algorithms will be examined in more detail as they present a proven method for deriving a practical result from a clustering operation and are already implemented as part of the API of MLlib.

Finally, it is worth considering that most forms of K-Means clustering require that the value of \texttt{K} is specified in advance. This is considered to be one of the main weaknesses of these algorithms as it is usually uncertain how many logical groups the data consists of. Similar issues with inconsistency can occur when poor initial values for the \texttt{K} centroids are chosen, as well as when the logical clusters, formed by the input data, are not approximately similar in size.

\section{Iterative K-Means Clustering}

TODO: citations for previous section

TODO: embedded pseudo-code

Iterative K-Means Clustering, like all clustering algorithms, is a form of unsupervised machine learning.

\begin{algorithm}
\caption{Iterative K-Means}\label{iterative}
\begin{algorithmic}[1]
\State
\end{algorithmic}
\end{algorithm}

\section{Mini-batch K-Means Clustering}

mini-batch text

\begin{algorithm}
\caption{Mini-batch K-Means}\label{mini-batch}
\begin{algorithmic}[1]
\State Given: \textit{k}, mini-batch size \textit{b}, iterations \textit{t}, data set \textit{X}
\State Initialize each \textbf{c} $\in$ \textit{C} with an \textbf{x} picked randomly from \textit{X}
\State v $\gets$ 0
\For{\textit{i} = 1 to \textit{t} }
    \State \textit{M} $\gets$ \textit{b} examples picked randomly from \textit{X} 
    \For {x $\in$ \textit{M}}
    	\State \textbf{d}[x] $\gets$ \textit{f}(\textit{C}, x) \hspace{1cm} // Cache the center nearest to \textbf{x}
    \EndFor
    \State \textbf{end for}
    \For {x $\in$ \textit{M}}
    	\State \textbf{c} $\gets$ \textbf{d}[x] \hspace{1.5cm} // Get cached center for this \textbf{x}
        \State \textbf{v}[\textbf{c}] $\gets$ \textbf{v}[\textbf{c}] + 1 \hspace{0.5cm} // Update per-center counts
        \State $\mu \gets \frac{1}{\textbf{v}[\textbf{c}]}$ \hspace{1.5cm} // Get per-center learning rate
        \State \textbf{c} $\gets (1 - \mu)$\textbf{c} + $\mu$\textbf{x} \hspace{1cm} //Take gradient step
   \EndFor
   \State \textbf{end for}
\EndFor
\State \textbf{end for}
\end{algorithmic}
\end{algorithm}

%==============================================================================

\chapter{Previous Work in MLlib}
\label{previous}

\section{Implementation of Iterative K-Means Clustering}

\section{Implementation of Mini-batch K-Means Clustering}

%==============================================================================

\chapter{Proposed Extension to Spark/MLlib}
\label{propose}

\section{Online K-Means}

\section{Adaptive Resonance Theory K-Means}

\section{Self-Organizing Maps K-Means}

\section{Benefits of Online Algorithms}

\section{Implementation Challenge in Parallelizing Sequential Algorithms}

%==============================================================================

\chapter{Online K-Means}
\label{online}

\section{Implementation Using Grand Mean}

\section{Implementation using Spark Streaming}

%==============================================================================

\chapter{Adaptive Resonance Theory K-Means}
\label{art}

\section{Implementation Using Spark Streaming}

%==============================================================================

\chapter{Self-Organizing Maps K-Means}
\label{som}

\section{Implementation Using Spark Streaming}

%==============================================================================

\chapter{Performance Evaluation and Data Model Analysis}
\label{eval}

maybe add an evaluation section earlier to each part\\
add chapter comparing algorithms at the end\\
both would need to explain evaluation method earlier\\
make a reference to the used training data

\section{Sum of Squared Distances}

\section{Silhouette}

%==============================================================================

\chapter{Conclusion}
\label{conclusion}

conclusion text - multiple sections?

%==============================================================================

\bibliographystyle{plain}
\bibliography{l4proj}

%==============================================================================

\begin{appendices}

\chapter{First Appendix}

\end{appendices}

\end{document}