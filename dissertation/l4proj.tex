\documentclass{l4proj}

\usepackage{url}
\usepackage{fancyvrb}
\usepackage[final]{pdfpages}
\usepackage{hyperref}

\hypersetup{
    colorlinks,
    citecolor=black,
    filecolor=black,
    linkcolor=black,
    urlcolor=black
}

\begin{document}
\title{Implementations of Advanced K-Means Clustering Algorithms \\ for Spark/MLlib}
\author{Ivan Kyosev}
\date{March 21, 2017}
\maketitle

\begin{abstract}
Apache Spark is a fast and general engine for large-scale data processing. It emphasizes speed, ease of use and generality in regards to working with Big Data. Its feature-set provides an alternative to the popular Hadoop MapReduce software framework, displaying a noticeable performance improvement. One of Spark's many components is MLlib. It is a machine learning library built on top of the core features of Spark and provides implementations for a variety of popular machine learning algorithms, written in a way meant to support scalability when operating on big data sets.

This paper presents a proposed extension to the MLlib component of Apache Spark in the form of additional K-Means Clustering algorithms that the library does not currently support. K-Means clustering is a method of vector quantization, originally from signal processing, that is popular for cluster analysis in data mining. A thorough analysis of the performance of the newly created algorithms and the relative quality of the produced data models is also conducted.
\end{abstract}

\educationalconsent

\tableofcontents

\chapter{Introduction}
\label{intro}
\pagenumbering{arabic}

Machine learning is an increasingly popular and relevant part of modern day systems looking to extract knowledge and value from the world around them. It refers to an approach to computation, where applications can learn without being explicitly programmed. These kinds of apps can analyze a provided sample of data and make predictions or decisions, based on some form of discovered pattern or correlation between the individual elements of the input set.

By being able to take rational choices, through examining the surrounding environment, systems that employ machine learning have become a central component of artificial intelligence. Because these system adapt their behavior based on new ``experiences'' they are utilized in a wide range of computing tasks, where developing explicit algorithms is not feasible. Such examples include solving problems in speech recognition, vision and robotics.

These machine learning algorithms are typically classified into three different categories based on the nature of the data they are learning from and the feedback of information for each input item:

\begin{itemize}
\item Supervised learning - in this case a program presented with a particular input set of data is also provided with the desired output, with the end goal of having the machine learn a mapping function from inputs to outputs
\item Unsupervised learning - here the elements of the input data set are not provided with a desired output value in pairs, so the system is left on its own to find structure in its input
\item Reinforcement learning - this refers to the idea of having an program interact with a dynamic environment, where it is attempting to achieve a certain goal. Along with every action, the program is also provided with positive or negative feedback, depending on whether the action aided in achieving the goal or not. 
\end{itemize}

Many of these machine learning algorithms have a similar underlying approach to their implementation. Input training data with some unknown probability distribution is presented and the program has to construct a model that allows it to extract meaning from the data and make sufficiently accurate predictions for instances that have not been encountered yet. This is generally achieved by defining a loss function and adjusting the model in attempts to minimize this loss with respect to the inputs -- effectively an optimization problem.

It is clear that machine learning algorithms heavily depend on the availability of input data in order to optimize their models to more accurately reflect the state of the real world. Therefore the quantity of supplied input data begins to have a significant impact on the quality of the output produced by application relying on machine learning. With the advent of big data and technologies meant to enable fast and reliable processing of big data, these algorithms can now make use of vast amounts of information, enabling them to create more accurate and valuable results. Examples of big data processing frameworks include Hadoop and Apache Spark.

The Apache Spark project in particular includes a component called MLlib. MLlib is a machine learning library built on top of the core features of Spark. It provides implementations for a variety of popular machine learning algorithms, all of which are written in a way meant to support horizontal scalability\footnote{Horizontal scaling refers to the scaling of a system by adding more parallel, similar in performance nodes. It is directly contrasted by vertical scaling where additional resources are added to the one node of a system.} when operating on big data sets.

\section{Aims}

The aim of this project is to create an extension to the MLlib component of Apache Spark, by developing implementations for a variety of K-Means clustering algorithms that the framework does not currently support. The newly introduced algorithms also need to be evaluated and contrasted with the ones already provided by Spark.

K-Means clustering is an example of an unsupervised machine learning algorithm. It is a method of vector quantization\footnote{Quantization is the process of mapping a large set of input values to a smaller set.}, that originated from signal processing. Its primary purpose is to group a collection of \texttt{N} data points into \texttt{K} distinct clusters. There is a variety of algorithms that produce the end result of grouping data points into \texttt{K} clusters, with varying levels of speed and accuracy. The ones that this project aims to implement as part of Spark could deliver a desired performance improvement with minimal loss in quality.

\section{Report outline}

The remainder of this report will focus on the mathematics underlying each of the considered K-Means clustering algorithms as well as the challenges of implementing them in a scalable fashion under Spark:
\begin{itemize}
\item \textbf{Chapter~\ref{spark}} covers the data processing model provided by Apache Spark and its machine learning library MLlib
\item \textbf{Chapter~\ref{kmeans}} explains the most common used versions of K-Means Clustering algorithms
\item \textbf{Chapter~\ref{previous}} discusses the approach taken by Spark's developers to create the current clustering API
\item \textbf{Chapter~\ref{propose}} outlines the proposed algorithms to be implemented as part of Apache Sparks, along with their benefits and challenges
\item \textbf{Chapter~\ref{online}} details two different approaches to the implementation of Lloyd's Online K-Means Clustering algorithm
\item \textbf{Chapter~\ref{art}} presents a version of K-Means using Adaptive Resonance Theory and its implementation
\item \textbf{Chapter~\ref{som}} presents the implementation of a  a K-Means Clustering algorithm using Self-Organizing Maps
\item \textbf{Chapter~\ref{eval}} covers a variety of ways to evaluate both the performance of the presented algorithms and the quality of the data models they produce
\item \textbf{Chapter~\ref{conclusion}} Concludes this report with a reflection on the whole project
\end{itemize}

%==============================================================================

\chapter{Introduction to Apache Spark and MLlib}
\label{spark}
\section{Apache Spark}

A new big data processing paradigm in the form of cluster computing has become widely popular in recent years. It is a model where data-parallel computations are executed on clusters of unreliable machines by systems that provide scheduling, fault tolerance and load balancing\cite{Spark}. Hadoop MapReduce was the first software framework to implement these principles\cite{MapReduce}, with systems like Dryad and Map-Reduce-Merge generalizing the supported types of data flows. These system achieve fault tolerance and scalability by supplying the user with a means of expressing computation as an acyclic data flow graph which passes the input data through a set of operations. This enables the underlying system to handle any occurring faults as well as scheduling without any user interaction.

Apache Spark is one of the latest big data processing frameworks which implements the cluster computing model. One of its primary advantages over Hadoop is that it can perform in-memory data processing, while Map-Reduce persists all outputs back to disk after a map or reduce action. This extends the scope of problems which can be solved efficiently by Spark to also include applications that reuse a working set of data across multiple parallel operations. It is impractical to implement these kinds of apps under Hadoop as between each iteration of an algorithm over the working set, the job must reload the data from disk, causing unnecessary overhead, resulting in degraded performance. Examples include iterative jobs and interactive analysis\cite{Spark}.

Due to its differences in design philosophy, users of the Spark framework have reported a 100x speed increase in certain work loads, versus a similar MapReduce solution, as well as a 10x faster execution on disk\cite{webSpark}. This performance improvement, however, results in the system requiring more RAM to carry out its operations.

The core components of Spark are implemented in Scala - a statically typed functional programming language that runs on top of the Java Virtual Machine. It also includes object-oriented features and supports Java interoperability thanks to its JVM roots -- this allows Spark to easily interact with Java based systems such as HDFS. In addition to Scala the framework also export an API available in Java, Python and R.

Spark can also be used interactively from a modified version of the Scala interpreter. This method provides full access to the API, allowing a user to define variables, functions, classes and run parallel jobs on a computer cluster. Spark is possibly the first system that allows a general-purpose programming language to be used interactively to process large data sets on a cluster\cite{Spark}.

\section{Resilient Distributed Datasets}

TODO: DACs

TODO: Transformation vs Actions

TODO: Narrow and Wide Dependencies

\section{Spark Streaming}

TODO: DStreams

\section{MLlib}

TODO: The MLlib paper

%==============================================================================

\chapter{K-Means Clustering}
\label{kmeans}

top level k-means text?

\section{Iterative K-Means Clustering}

iterative k-means text

\section{Mini-batch K-Means Clustering}

mini-batch text

%==============================================================================

\chapter{Previous Work in MLlib}
\label{previous}

\section{Implementation of Iterative K-Means Clustering}

\section{Implementation of Mini-batch K-Means Clustering}

%==============================================================================

\chapter{Proposed Extension to Spark/MLlib}
\label{propose}

\section{Online K-Means}

\section{Adaptive Resonance Theory K-Means}

\section{Self-Organizing Maps K-Means}

\section{Benefits of Online Algorithms}

\section{Implementation Challenge in Parallelizing Sequential Algorithms}

%==============================================================================

\chapter{Online K-Means}
\label{online}

\section{Implementation Using Grand Mean}

\section{Implementation using Spark Streaming}

%==============================================================================

\chapter{Adaptive Resonance Theory K-Means}
\label{art}

\section{Implementation Using Spark Streaming}

%==============================================================================

\chapter{Self-Organizing Maps K-Means}
\label{som}

\section{Implementation Using Spark Streaming}

%==============================================================================

\chapter{Performance Evaluation and Data Model Analysis}
\label{eval}

maybe add an evaluation section earlier to each part\\
add chapter comparing algorithms at the end\\
both would need to explain evaluation method earlier\\
make a reference to the used training data

\section{Sum of Squared Distances}

\section{Silhouette}

%==============================================================================

\chapter{Conclusion}
\label{conclusion}

conclusion text - multiple sections?

%==============================================================================

\bibliographystyle{plain}
\bibliography{l4proj}

%==============================================================================

\begin{appendices}

\chapter{First Appendix}

\end{appendices}

\end{document}